\chapter{Conclusiones y trabajo futuro}
\label{cap:trabajoFuturo}

\chapterquote{Haz lo que creas que es interesante; haz algo que creas que es divertido y que merezca la pena, porque si no, no lo vas a hacer bien de todas formas.}{Brian Kernighan}


\begin{resumen}
En este último capítulo se resume brevemente todo el trabajo realizado, mencionando también cualquier trabajo que puede ser interesante para el futuro del proyecto.
\end{resumen}

Gracias al desarrollo de \baker{} se ha conseguido implementar una herramienta útil de cara a la creación de videojuegos 2D de estilo RPG. Aún siendo un proyecto con menor alcance a otros motores específicos del género, es capaz de generar juegos muy completos cubriendo muchas de las necesidades que estos requieren. Con él se acerca a un público sin experiencia en programación a la creación de este tipo de juegos de forma libre, abriendo las puertas a que una mayor cantidad de gente se pueda dedicar a ello. Una de las fortalezas de \baker{} es su enfoque multiplataforma, con un editor funcional en Windows, MacOS y Linux, lo que facilita su uso en los entornos de desarrollo más comunes y ofrece una mayor comodidad a los usuarios sin importar su sistema operativo preferido. Por otra parte, el motor funciona en Windows, MacOS, Linux y también en Android, ampliando así notablemente el alcance de los juegos publicados, especialmente al incluir soporte para uno de los mercados más importantes en la industria actual, como lo es el de los dispositivos móviles.

\medskip  

Con todo esto, aún cumpliendo todos nuestros objetivos propuestos, \baker{} podría llegar a ofrecer más herramientas que permitan añadir otros sistemas propios de los RPG. Tras haber obtenido una gran cantidad de \textit{feedback} en las pruebas de usuario, principalmente por usuarios más experimentados en el uso de herramientas como \textit{RPG Maker}, se han dejado como posibles ampliaciones los siguientes elementos:
\begin{itemize}
	\item Sistema de combate: Al tener típicamente esta clase de juegos enfrentamientos entre personajes, sería un muy buen añadido el permitir crear y personalizar estas batallas. Idealmente se permitiría personalizar, lo primero, a aquellos personajes que vayan a combatir, definiendo sus estadísticas, sus formas de ataque y las maneras en las que estos progresarían tanto del lado del jugador como de los enemigos. La segunda parte esencial para conformar este apartado sería la creación de escenas de combate. Siendo los enfrentamientos por turnos típicos en los RPG encajaría perfectamente en el entorno, se podría personalizar qué personajes se enfrentarían, en qué momento se escogerían las acciones de un turno y en qué orden actuarían los combatiente.
	\item Sistema de \textit{items}: Otra parte clásica de los RPG son los \textit{items}, objetos que puede guardar un jugador, como armas o pociones. Estos irían asociados a un inventario donde guardarlos que podría tener el jugador. Estos objetos aplicarían efectos que modificarían el estado de los personajes o el juego. Cada uno de ellos tendría alguno de estos efectos asociados que podrían aplicarse tanto navegando el mundo como en los combates dependiendo del caso.
\end{itemize}

\medskip 

En resumen, nuestro editor permite de forma accesible crear, con el motor, videojuegos RPG sencillos en los que navegar por un mundo de manera dinámica. Sobre esto, aún siendo ya una herramienta potente, se podrían hacer añadidos que ofrezcan una experiencia más amplia para crear RPG. Permitiendo así a aquella gente menos experimentada en el desarrollo construir juegos aún más completos.