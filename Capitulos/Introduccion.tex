\chapter{Introducción}
\label{cap:introduccion}

\chapterquote{Una vez tuve una conversación bastante rara con un par de abogados y estaban hablando sobre: \comillas{¿Cómo elegís a vuestro público objetivo? ¿Hacéis ``focus groups'', encuestáis a gente y todo eso?} Y es como: \comillas{No, simplemente hacemos juegos que creemos que molan.}}{John Carmack}

\section{Motivación}
Los videojuegos de rol han sido uno de los géneros más influyentes de la industria, desde sus orígenes en la década de los años 80 hasta la actualidad, donde concentran una gran parte de todos los juegos vendidos a diario.

\smallskip

El desarrollo de este tipo de juegos ha sufrido cambios mayúsculos con el paso de los años y con las consecuentes mejoras \textit{hardware} y \textit{software}, que nos han permitido evolucionar desde máquinas diseñadas exclusivamente para poder ejecutar un único juego a la amplia gama de dispositivos multimedia de los que disponemos actualmente.

\medskip

Numerosos programas de edición y desarrollo de videojuegos han aparecido en las últimas dos décadas, pero aquellos que son más comerciales están pensados para juegos de todo tipo, por lo que suelen tener características más generales y no tan estrechamente relacionadas con el desarrollo de juegos de rol, y muchas veces restringen algunas de sus funcionalidades, lo que dificulta el desarrollo.

\smallskip

Aquellos programas que sí que lo están tienen dos problemas fundamentales:
\begin{itemize}
	\item Los que ofrecen una interfaz intuitiva, sencilla de utilizar, y bastante amigable con nuevos usuarios que están introduciéndose en el mundo del desarrollo de videojuegos están bajo un muro de pago, que pese a no ser muy elevado, hace que usuarios aficionados paguen por un \textit{software} que rara vez utilizarán si no se afianzan finalmente en el mundo del desarrollo.
	\item Los que no están bajo un muro de pago, es decir, son \textit{software} de código libre, suelen tener interfaces y sistemas complicados de entender para noveles, quienes debido a la complejidad de estas herramientas deciden abandonar por completo el desarrollo.
\end{itemize}

\section{Objetivos}
Descripción de los objetivos del trabajo.

\section{Plan de trabajo}
Aquí se describe el plan de trabajo a seguir para la consecución de los objetivos descritos en el apartado anterior.

\section{Herramientas y Metodología}
Descripción de las herramientas utilizadas en el trabajo y de la metodología seguida para llevar a cabo las etapas del plan de trabajo.

\section{Estructura de la memoria} 
Explicación breve de la estructura de la memoria (qué se encuentra en cada parte) y algunos enlaces de interés al repositorio con el proyecto, etc...

\section{Enlaces adicionales}