\chapter{Evaluación y análisis}
\label{cap:conclusiones}

\chapterquote{\comillas{¿Cómo hacer pruebas?} es una pregunta que no puede responderse en general. Sin embargo, \comillas{¿cuándo se hacen las pruebas?} sí que tiene una respuesta: tan pronto y tan frecuentemente como sea posible.}{Bjarne Stroustrup}


\begin{resumen}
En este capítulo se explicarán las pruebas con usuarios realizadas, así como la metodología seguida en las pruebas y unas pequeñas conclusiones sobre estas.
\end{resumen}

Al conocer en profundidad cómo funcionan todos los sistemas del motor y el editor, no se puede tener la certeza de que estos funcionen como un usuario promedio espera. Por esto, realizar pruebas con usuarios sin sesgos es crucial para obtener la información de las posibles brechas entre los programas y esos usuarios. De cara a ofrecer una experiencia de uso agradable se analizarán las pruebas en busca de formas de mejorarla. Se buscarán distintos tipos de perfiles para obtener una mayor cantidad de datos sobre los posibles conflictos que estos tengan con los sistemas.

\section{Objetivo de la evaluación}
Los objetivos de las pruebas con usuarios son principalmente dos. El primero es evaluar si los usuarios entienden los sistemas y son capaces de usarlos cómodamente. Es importante que el entorno sea agradable de usar para permitir que los usuarios entiendan las capacidades del mismo y las puedan aprovechar. El segundo es evaluar si cada usuario es capaz de usar el editor de forma creativa, de tal manera que los videojuegos que cree un usuario supongan una experiencia diferente a los que haga otro, comprobando así la flexibilidad de uso del editor.

\section{Metodología}
Para llevar a cabo estas pruebas, una vez definidos los objetivos, se diseñó el procedimiento a seguir. Cada usuario realizará las pruebas de manera individual, acompañado por un supervisor, y con una duración estimada de una hora.

\smallskip

Al inicio, se le pregunta a cada probador sobre su experiencia tanto en jugar como en crear videojuegos, haciendo especial énfasis en los RPG. Con esta información recopilada, se le entregan varios archivos: el ejecutable del editor, un paquete de recursos para evitar que se pierda tiempo buscando durante la prueba, y una breve guía de uso del editor.

\smallskip

Esta guía, disponible en el apéndice \ref{Appendix:Key1}, explica de forma concisa cómo utilizar cada uno de los sistemas disponibles en el editor. Con esta información, el probador seguirá la guía para entender los mecanismos y, posteriormente, podrá usarlos libremente según su criterio.

\medskip

El investigador será responsable de registrar todas las interacciones relevantes de los probadores con el editor, además de brindar asistencia en caso de que surja algún error durante las pruebas. Para observar una experiencia lo más real posible, el investigador no intervendrá en las acciones del usuario, salvo en situaciones excepcionales, como errores del programa o bloqueos en los que el probador no pueda continuar debido a alguna dificultad.

\smallskip

Estas pruebas están diseñadas para realizarse tanto de manera presencial como telemática, adaptándose a la mayor comodidad de cada caso particular. Al finalizar la prueba, se solicitará al usuario una valoración general sobre los sistemas y su experiencia de uso, incluyendo sus quejas e ideas.

\section{Resultados}

En total, las pruebas han sido realizadas con quince personas, las cuales podemos dividir en tres categorías según su experiencia con los videojuegos y, en particular, con los RPG.

\smallskip 

En primer lugar, cinco de los usuarios que son los menos familiarizados con el mundo de los videojuegos, con poca experiencia jugando en general y, en particular, con los juegos RPG. Con este tipo de usuario, el objetivo es evaluar qué tan intuitivas son las herramientas y la interfaz de usuario del editor, así como comprobar si se entienden bien las instrucciones de las pruebas. Este primer grupo logró crear mapas con cierto sentido, pero la mayoría utilizó únicamente una o dos capas, lo que limitó un poco el diseño. No dedicaron mucho tiempo a explorar los \textit{tilesets} y realizaron diseños bastante sencillos. Las configuraciones las entendieron bastante bien y, siguiendo la guía, lograron completarlas. En la creación de eventos se quedaron en los comportamientos más simples y no intentaron crear eventos complejos usando variables y condiciones. En general, la prueba fue satisfactoria y los resultados estuvieron dentro de lo esperado.

\medskip

En segundo lugar, siete de los usuarios que sí juegan videojuegos y algunos incluso están relacionados con la industria, pero no tienen conocimientos de programación o desarrollo más allá de sus áreas específicas. Estos usuarios aprovecharon mucho más el apartado de \textit{mapeado}, utilizando más capas y varios \textit{tilesets} para crear mapas mucho más interesantes. En cuanto a los eventos, se les animó más que al primer grupo a intentar crear eventos algo más complejos, pero al parecer no tenían claro cómo estructurarlos para aprovechar al máximo las funcionalidades, y la mayoría se quedó en cosas más sencillas o requirió bastante ayuda por parte de los supervisores. Aunque los eventos no requieren conocimientos de programación, su estructura es similar a la de un sistema basado en lógica condicional, lo que exige más tiempo del disponible durante la prueba para entender cómo organizarlos correctamente.

\medskip

Por último, está el grupo de tres usuarios con conocimientos en el desarrollo de videojuegos, incluso en el desarrollo específico en \textit{RPG Maker}, lo cual fue muy útil para obtener \textit{feedback} comparativo. Este grupo comprendió bien las herramientas de \textit{mapeado} y señaló que echaba en falta algunas opciones más avanzadas que agilizaran el proceso, como la posibilidad de seleccionar varios \textit{tiles} a la vez o rellenar áreas con \textit{tiles} de un mismo tipo. También comentaron que el proceso de creación de \textit{sprites} y animaciones es lento, proponiendo algún tipo de automatización similar a la que existe para los \textit{tilesets}. En cuanto a los eventos, lograron entender cómo montar estructuras más avanzadas, lo cual fue un resultado positivo.

\medskip

Adicionalmente a todas las observaciones y el \textit{feedback} recibido, las pruebas también sirvieron para detectar varios \textit{bugs} y problemas de funcionamiento, lo que fue de gran ayuda para dar un pulido final a la versión del editor.

\section{Análisis de los resultados y conclusiones}
Aunque las pruebas no alcanzaron la extensión ni la variedad deseadas debido a limitaciones logísticas, se pueden extraer algunas conclusiones interesantes para el desarrollo actual y, sobre todo, para un posible trabajo futuro.

\smallskip 

Una de las prioridades identificadas es mejorar la experiencia de diseño de mapas y la creación de \textit{sprites} y animaciones. Con este objetivo, se proponen las siguientes mejoras funcionales:

\begin{itemize}
	\item Incorporar la posibilidad de seleccionar varios \textit{tiles} simultáneamente y pintarlos en el mapa de forma conjunta. 

	\item Añadir una herramienta para rellenar automáticamente una zona completa con el \textit{tile} seleccionado. 

	\item Permitir la selección de \textit{tiles} directamente desde el mapa, sin necesidad de localizarlos manualmente en el \textit{tileset}. 

	\item Incluir una función para deshacer la última edición realizada. 

	\item Automatizar el proceso de creación de múltiples \textit{sprites} y su conversión en animaciones a partir de una única fuente, de manera similar al funcionamiento del editor de \textit{tilesets}. 
\end{itemize}

Además, se considera conveniente implementar mejoras en forma de \textit{tooltips} en la sección de eventos, con el objetivo de explicar de manera clara todas las funcionalidades disponibles. También se sugiere incluir eventos predefinidos que sirvan como ejemplos prácticos para facilitar su comprensión y uso.

