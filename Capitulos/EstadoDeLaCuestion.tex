\chapter{Estado de la cuestión}
\label{cap:estadoDeLaCuestion}

\begin{resumen}
En este capítulo vamos a hablar sobre los videojuegos de rol, introducir un poco su historia, hablar sobre el desarrollo de videojuegos, qué es un motor, para qué sirve, qué lo compone, qué es un editor, para qué sirve, y cuáles son los editores especializados en el desarrollo de videojuegos de rol.
\end{resumen}

\section{Sobre los videojuegos de rol}
Para poder definir qué es un videojuego de rol necesitaremos primero entender \textit{qué es un videojuego} y \textit{qué es un juego de rol}.

\medskip

Definir qué es un videojuego es una tarea bastante complicada, sobre todo teniendo en cuenta los matices con los que podemos definirlo (académicos, de diseño, experimentales o tecnológicos). Una de las definiciones más acertadas nos la da \cite{EspositoVJ}, que lo define como \comillas{un \textit{juego} que se \textit{juega} gracias a un \textit{aparato audiovisual}, y que puede estar basado en una \textit{historia}}. El propio \citeauthor{EspositoVJ} se encarga de definir qué es el \textit{juego}, qué es \textit{jugar}, qué es el \textit{aparato audiovisual} y qué es la \textit{historia}. La diferencia fundamental de los juegos tradicionales frente a los videojuegos es la existencia de ese \textit{aparato audiovisual} (videoconsolas, ordenador, dispositivos móviles) que pueda ofrecer una interacción \comillas{humano-máquina}, ya que es esta interacción recíproca la que hace que los videojuegos sean videojuegos y no otro tipo de entretenimiento multimedia.

\medskip

Los juegos de rol, al haber nacido desde juegos de mesa tradicionales, han tenido el suficiente tiempo como para poder desarrollarse y poder ajustarse a una definición mucho más precisa, aunque, nuevamente, debido a la amplia variedad de juegos que se pueden catalogar como \comillas{de rol}, es posible que hasta la definición más completa no pueda abarcarlos a todos. \citeauthor{LortzRPG} (\citeyear{LortzRPG}, citado en \cite{FineRPG}), define los juegos de rol como \comillas{todos aquellos juegos que permiten a un determinado número de jugadores asumir los roles de personajes imaginarios y operar con cierto grado de libertad en un entorno imaginario}.

\smallskip

Por otra parte, \citeauthor{TychsenRPG} (\citeyear{TychsenRPG}, citado en \citet*{RPGH&D}) enumera los elementos imprescindibles que un juego debe tener para ser considerado como juego de rol:
\begin{itemize}
	\item Narrar una historia adecuándose a una serie de reglas. Tanto la historia como las reglas son únicas para cada juego.
	\item Reglas, múltiples participantes (al menos dos) y un mundo ficticio sobre el que se va a desarrollar la acción. Todos los participantes deben conocer previamente la ambientación, el entorno y las reglas.
	\item La gran mayoría de participantes controlarán, como mínimo, a un personaje durante toda la partida, y será con ese o esos personajes con quienes interactuará con el entorno.
	\item La existencia, por lo general, de un \textit{director de juego} (GM, \textit{gamemaster}), que será el responsable de gestionar aquellos elementos del juego o del entorno ficticio que no se encuentran bajo el control directo de los jugadores.
\end{itemize}

\medskip

Ahora que ya sabemos \textit{qué es un videojuego} y \textit{qué es un juego de rol}, y pese a que no podemos dar una definición válida para todos los videojuegos que estén en esta categoría, definiremos un videojuego de rol (RPG, \textit{role-playing game}, o también, más raramente, CRPG, \textit{computer role-playing game}) como todo aquel programa \textit{software}, con carácter lúdico, que permita a sus usuarios encarnar el rol de uno o varios personajes en un mundo ficticio, donde pueden tomar decisiones, interactuar con su entorno o mundo con cierta libertad y desarrollar una narrativa para conseguir un determinado objetivo.

\smallskip

Como se ha mencionado anteriormente, esta definición vale para la gran mayoría de RPG; pero, si pensamos en los videojuegos que se venden cada día en las tiendas, una proporción significativa de estos puede encajar dentro de la definición anterior sin necesariamente tener que ser un RPG, ya que prácticamente en todos se encarna el rol de un personaje, casi todos permiten una interacción con el entorno y todos tienen un objetivo final. Esto nos lleva a plantearnos si las clasificaciones de videojuegos según su género son demasiado limitadas para los videojuegos actuales.

\subsection{El problema de los géneros de videojuegos}
Antes de adentrarnos en los RPG, veremos por qué las distintas clasificaciones de videojuegos atendiendo a características similares tienen bastantes lagunas a la hora de categorizar las entregas más modernas.

\medskip

Los primeros intentos de clasificar los videojuegos mediante características comunes surgen en la década de los 80, principalmente por diseñadores y desarrolladores, como \cite{Crawford84}, quien los categorizó entre aquellos que \textit{requieren de habilidades previas por parte del usuario} (como juegos de combate o de carreras), y \textit{juegos de estrategia} (que engloba al resto de juegos, como los de aventuras, los educativos, y los RPG). Estas categorías son \textit{funcionales}, se centran en las mecánicas de juego y enfatizan cómo los jugadores interactúan con el sistema.

\smallskip

En la actualidad, los distintos géneros vienen dados por una mezcla de mecánicas, temas, elementos narrativos, estética, lugar de origen y plataforma, y están gravemente influenciados por las tendencias del mercado, discursos mediáticos y la propia percepción de los jugadores. También, muchas veces se quiere categorizar a los videojuegos con etiquetas propias de otras modalidades, como el cine o la literatura, que son incapaces de capturar los aspectos únicos que definen a un videojuego.

\smallskip

\cite{FailGeneros} argumentan que las categorías existentes a día de hoy se quedan cortas para satisfacer los propósitos del género en videojuegos (identidad, agrupación, \textit{marketing} y educación), ya que dada la gran diversidad de juegos que tenemos ahora, resulta casi imposible poder agrupar la naturaleza multifacética de estos en etiquetas tan \comillas{tradicionales}.

\smallskip

Pongamos el ejemplo de uno de los juegos más exitosos de la última década, \citegame{The Legend of Zelda: Breath of the Wild}{BOTW}, que mezcla elementos de libre exploración (lo que lo convertiría en un \textit{sandbox} o \comillas{juego libre}), acción en tiempo real e investigación y resolución de rompecabezas (lo que lo convertiría en un juego de \comillas{acción-aventura}), y la progresión en tiempo real del personaje característica de los RPG (puedes ir consiguiendo nuevas habilidades o mejorando el equipamiento). Es por eso que muchas veces tenemos géneros intermedios, como en este caso, que podríamos considerar a \textit{Breath of the Wild} como un RPG de acción (ARPG, \textit{action role-playing game}, que igualmente siguen sin englobar a la increíble diversidad de juegos. Este problema también se puede aplicar a la inversa, donde juegos como \citegame{Undertale}{Undertale}, esencialmente un RPG, tiene elementos, como el combate, propios de otro género de juegos.

\smallskip

Hay muchas formas de evitar este problema, y quizá la mejor solución sea dejar de considerar a los géneros como \comillas{cajones estancos} en los que un videojuego no pueda pertenecer a dos de estas categorías simultáneamente \citep{Apperley}, sino que sean más bien un espectro, sin fronteras establecidas, en el que un videojuego pueda caer entre dos categorías distintas.

\smallskip

En resumen, antes de categorizar un videojuego en según qué género, hay que entender que resulta imposible que este se reduzca a una fórmula fija, sino que hay que entenderlo como una combinación fluida de mecánicas, elementos narrativos, temas y estética, que varían de juego a juego.

\subsection{Historia de los videojuegos de rol}
Es a mediados de la década de los 70 cuando podemos hablar del nacimiento de los videojuegos de rol. Dadas las limitaciones tecnológicas de la época, los RPG primitivos no eran más que simples adaptaciones de juegos de mesa ya existentes por entonces, como \citegame{dnd}{dnd}, una adaptación de \citegame{Dungeons \& Dragons}{ogdnd}, quizás el juego de rol más famoso, que mezcla mecánicas de combate con temas fantásticos. Estos primeros juegos contaban con interfaces basadas en texto, y normalmente estaban programados en los grandes ordenadores que se encontraban en campus universitarios como los de Harvard o la Universidad de Illinois.

\smallskip

Es en la década de los 80, y, gracias a los avances tecnológicos, que los RPG comienzan a separase cada vez más de ser meras adaptaciones de juegos ya existentes a desarrollar sus propias historias. A partir de esta época encontramos dos corrientes bastante diferenciadas de RPG, los \comillas{occidentales} (WRPG, \textit{Western role-playing game}), con más libertad de decisión para los jugadores tanto en personalización como en la historia, y con temáticas realistas; y los \comillas{orientales} o \comillas{japoneses} (JRPG, \textit{Japanese role-playing game}, por ser Japón el país que más videojuegos de este tipo produce), centrados en una narrativa lineal con temáticas fantásticas y mecánicas basadas por turnos.

\section{Sobre el desarrollo de videojuegos}

\subsection{Motor de videojuegos}

\subsubsection{Componentes de un motor de videojuegos}

\subsubsection{Separación entre motor y \textit{gameplay}}

\subsubsection{Programación dirigida por datos (\textit{DDP}) en videojuegos}

\subsubsection{Programación multiplataforma}

\subsection{Editor de videojuegos}

\subsubsection{Editores específicos para desarrollo de \textit{RPG}}

