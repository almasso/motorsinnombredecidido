\chapter{Estado de la cuestión}
\label{cap:estadoDeLaCuestion}

En el estado de la cuestión es donde aparecen gran parte de las referencias bibliográficas del trabajo. Una de las formas más cómodas de gestionar la bibliografía en {\LaTeX} es utilizando \textbf{bibtex}. Las entradas bibliográficas deben estar en un fichero con extensión \textit{.bib} (con esta plantilla se proporciona el fichero biblio.bib, donde están las entradas referenciadas más abajo). Cada entrada bibliográfica tiene una clave que permite referenciarla desde cualquier parte del texto con los siguiente comandos:

\begin{itemize}
\item Referencia bibliografica con cite: \cite{ldesc2e}
\item Referencia bibliográfica con citep: \citep{notsoshort}
\item Referencia bibliográfica con citet: \citet{latexAPrimer}
\end{itemize}

Es posible citar más de una fuente, como por ejemplo \citep{latexCompanion,LaTeXLamport,texKnuth}

Después, \LaTeX se ocupa de rellenar la sección de bibliografía con las entradas \textbf{que hayan sido citadas} (es decir, no con todas las entradas que hay en el .bib, sino sólo con aquellas que se hayan citado en alguna parte del texto).

Bibtex es un programa separado de latex, pdflatex o cualquier otra cosa que se use para compilar los .tex, de manera que para que se rellene correctamente la sección de bibliografía es necesario compilar primero el trabajo (a veces es necesario compilarlo dos veces), compilar después con bibtex, y volver a compilar otra vez el trabajo (de nuevo, puede ser necesario compilarlo dos veces).

\section{Sobre los videojuegos de rol}
Los videojuegos son \textit{juegos} que se \textit{juegan} gracias a un \textit{aparato audiovisual} y que pueden estar basados en una \textit{historia} \citep{EspositoVJ}. Esposito define muy bien qué es el \textit{juego}, qué es \textit{jugar}, qué es el \textit{aparato audiovisual} y qué es la \textit{historia}. De esta definición nos interesa la parte de \textit{juego}, ya que es la que más vinculación tiene con el rol.

\medskip

Un juego de rol son todos aquellos juegos que permiten a un determinado número de jugadores asumir los roles de personajes imaginarios y operar con cierto grado de libertad en un entorno imaginario (Lortz (1979) citado en \cite{FineRPG}).

\medskip

Por tanto, atendiendo a ambas definiciones, definimos un videojuego de rol (\acl{RPG}) como todo aquel programa \textit{software}, con carácter lúdico, que permita a sus usuarios encarnar el rol de uno o varios personajes y, mediante estos, poder interactuar con su entorno o mundo con cierta libertad para conseguir un determinado objetivo.

\subsection{Historia de los videojuegos de rol}


\section{Sobre el desarrollo de videojuegos}

\subsection{Motor de videojuegos}

\subsubsection{Componentes de un motor de videojuegos}

\subsubsection{Separación entre motor y \textit{gameplay}}

\subsubsection{Programación dirigida por datos (\textit{DDP}) en videojuegos}

\subsubsection{Programación multiplataforma}

\subsection{Editor de videojuegos}

\subsubsection{Editores específicos para desarrollo de \textit{RPG}}

