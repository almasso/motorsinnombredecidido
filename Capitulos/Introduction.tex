\chapter*{Introduction}
\label{cap:introduction}
\addcontentsline{toc}{chapter}{Introduction}

\chapterquote{I had this really bizarre conversation once with a couple of lawyers and they were talking about \comillas{How do you pick your target market? Do you use focus groups and poll people and all this?} It's like \comillas{No, we just write games that we think are cool.}}{John Carmack}


\section*{Motivation}
Role-playing video games are one of the most influential genres in the industry\footnote{In total, more than one billion copies of role-playing video games have been sold throughout history, according to \textit{VGChartz} (\url{https://shorturl.at/VI2Zy}).}, from their origins in the 1980s to the present day, where they account for a large portion of the market share\footnote{According to \textit{Rocket Brush Studio}, role-playing video games correspond to the best-selling genre in the mobile market, the third-best in the PC market, and the fifth-best in the console market (\url{https://rocketbrush.com/blog/most-popular-video-game-genres-in-2024-revenue-statistics-genres-overview}).}.

\smallskip

The development of this type of game has undergone significant changes over the years, along with consequent improvements in hardware and software, which have allowed them to evolve from simple monochrome scenes with text-based interfaces to real-time 3D graphics, complex open worlds, and advanced interaction systems. These improvements have influenced not only the visual aspect but also the narrative, gameplay, and player immersion, enabling increasingly rich and personalized experiences.

\medskip

Numerous video game development and editing tools have appeared in the last two decades, but the most commonly used ones are designed for games of all kinds. As a result, they tend to offer general features that are not closely related to role-playing game development, and they often restrict certain functionality, making the development process more complicated.

\smallskip

Those tools that are specifically designed for role-playing game development have two fundamental flaws:
\begin{itemize}
	\item Those that offer an intuitive, easy-to-use interface that is friendly to newcomers in game development are locked behind a paywall which, although not very expensive, forces amateur users to pay for a software they will rarely unless they consolidate as developers.
	\item Those that are not locked behind a paywall or are open source, often have interfaces and systems that are difficult for beginners to understand, leading many to abandon game development due to the complexity of these tools.
\end{itemize}

\medskip

The main motivation behind this project arises from the need to have easy-to-use and flexible tools, both for experimented independent developers who desire to create games without the limitations imposed by general-purpose engines, and for people with little knowledge of video game development or programming.

\section*{Objectives}
This bachelor's thesis aims to develop \baker, a 2D role-playing game development environment consisting of a cross-platform engine capable of running on Windows, MacOS, Linux and Android, along with an editor, also cross-platform, that works on Windows, MacOS and Linux, and enables simple and straightforward game development for this engine. The editor will be capable of generating executable files that users can distribute without the need for any additional steps after development.

\smallskip

The editor's interface will be designed for beginners in game development, while still alowing more experienced users to create larger and more ambitious projects.

\medskip

To accomplish this:
\begin{itemize}
	\item A cross-platform (on Windows, MacOS, Linux and Android) engine that allows the user to implement role-playing games will be developed.
	\item A cross-platform (on Windows, MacOS and Linux) editor that eases the implementation of a role-playing game in the developed engine will be developed.
	\item \baker{} will be tested with users to demonstrate its functionality, and necessary conclusions will be drawn, along with potential improvements for the future.
\end{itemize}

\section*{Work Plan}
\begin{figure}[t]
	\scalebox{0.9}{
		\begin{ganttchart}[
    		hgrid,
    		vgrid,
    		title label font=\small,
    		title height=1.2,
    		x unit=0.65cm
  			]{1}{16}
  			
  			\gantttitle[title/.style={fill=blue!30, draw=black, line width=0.4pt}]{2024}{6}
  			\gantttitle[title/.style={fill=green!30, draw=black, line width=0.4pt}]{2025}{10}\\
  			
  			\gantttitle[title/.style={fill=blue!15, draw=black, line width=0.4pt}]{Oct.}{2}
  			\gantttitle[title/.style={fill=blue!15, draw=black, line width=0.4pt}]{Nov.}{2}
  			\gantttitle[title/.style={fill=blue!15, draw=black, line width=0.4pt}]{Dec.}{2}
  			\gantttitle[title/.style={fill=green!15, draw=black, line width=0.4pt}]{Jan.}{2}
  			\gantttitle[title/.style={fill=green!15, draw=black, line width=0.4pt}]{Feb.}{2}
  			\gantttitle[title/.style={fill=green!15, draw=black, line width=0.4pt}]{Mar.}{2}
  			\gantttitle[title/.style={fill=green!15, draw=black, line width=0.4pt}]{Apr.}{2}
  			\gantttitle[title/.style={fill=green!15, draw=black, line width=0.4pt}]{May}{2}\\
	
			\ganttbar{Research on the current state}{1}{9}\\
			\ganttbar{\baker{} design}{1}{9}\\
			\ganttbar{Engine development}{9}{16}\\
			\ganttbar{Editor development}{9}{16}\\
			\ganttbar{Document writing}{11}{16}\\
			\ganttbar{User testing}{13}{15}
		\end{ganttchart}
	}
	\caption{Gantt diagram showing the temporal planning of the work.} 
	\label{fig:ganttingles}
\end{figure}

To accomplish the previous objectives, the work plan will be divided in four phases: 
\begin{itemize}
	\item Research on the current state of engines and editors specifically designed for role-playing games, as well as on role-playing games themselves. This section will aim to identify the common features shared by all engines, editors and games, both open-source and commercial ones, and propose improvements to address the issues these tools may present for inexperienced new users. 
	\item \baker{} design. Based on the common features identified in the previous phase, an initial design will be proposed to serve as a basis for the development of the project. This design, while not immutable, should be as close to final as possible to avoid any issues during the development phase.
	\item \baker{} development. Once the design phase is completed, development will begin. This phase will, in turn, be split across multiple phases:
		\begin{itemize}
			\item Setting up the development environment. A development environment will be chosen based on both applications' needs, and everything will be configured to minimize effort during the development.
			\item Engine development. An engine will be developed according to the previously established plans, with cross-platform support for both PCs and Android mobile devices.
			\item Editor development. Like the engine, the editor will be developed according to the established design. It must support multiple platforms on PC, namely Windows, MacOS and Linux, but not Android.
		\end{itemize}
	\item User testing. To ensure the correct functioning of \baker, the final tools will be tested by a variety of users not involved in their development. After the tests are completed, the necessary conclusions will be drawn, and the tools will be adjusted to address any critical issues that require attention before releasing the public version. Features that would require excessive effort to implement within the available time will be left as future work.
\end{itemize}

Figure \ref{fig:ganttingles} shows a Gantt chart illustrating the project timeline, which began in October 2024 and ended in May 2025. The research and design phase took place during the first part of the project, from its inception until mid-February; the implementation phase spanned from mid-February to the end of May; the writing of this document was carried out from early March to late May; and finally, user testing was conducted between the second half of April and the first half of May.


\section*{Tools and Methodology}
Regarding the tools, Git will be used as the version control system, with a repository hosted on GitHub and managed through GitHub Desktop. Task management will be carried out using GitHub Projects, available on the GitHub website.

\smallskip

Access to the repository containing the code may be done through this link: \href{https://github.com/almasso/rpgbaker}{\textbf{https://github.com/almasso/rpgbaker}}.

\medskip

Regarding the development tools for \baker, CLion will be used as the C++ development environment, and Android Studio for Android development in Java. CMake will be used as the tool for generating external libraries, and MinGW for handling compilation on Windows, Clang on MacOS, and GCC on Linux. Detailed reasons for choosing these tools can be found in section \ref{sec:decisiones}.

\medskip

On the other hand, PDF generation of the document will be carried out using \LaTeX, with the \texis{} template, and Texmaker will be used as the editing environment.

\bigskip

Regarding the working methodology, bi-weekly meetings will be proposed with the tutor, although the number of weeks may vary depending on the progress made. During these meetings, the current status of the project will be presented, and advice related to the design or development will be requested.

\medskip

Communication with the team will be established through both in-person meetings when important issues need to be addressed and via software that allows messaging and voice chat, such as Discord. This tool will also be used for conducting user testing.

\section*{Document Structure} 
In Chapter \ref{cap:juegosrol}, \textit{Role-playing games}, the topic of role-playing games is discussed, including their characteristics, origins, and an analysis of some of the most important titles in the genre.

\medskip

In Chapter \ref{cap:videojuegosrol}, \textit{Role-playing video games}, role-playing video games and their history are introduced, along with the development process and the use of game engines and editors.

\medskip

In Chapter \ref{cap:planteamiento}, \textit{Project planning}, the design and decisions made prior to the development of the applications are discussed in detail.

\medskip

In Chapter \ref{cap:motor}, \textit{Engine}, the implementation of \baker's engine is discussed, along with some of the decisions made regarding its development.
\medskip

In Chapter \ref{cap:editor}, \textit{Editor}, the implementation of \baker's editor is discussed, along with some of the decisions made regarding its development.

\medskip

In Chapter \ref{cap:conclusiones}, \textit{Evaluation and analysis}, the research questions and objectives are presented, along with the development of the user testing phase and an analysis of the tests, from which conclusions are drawn.

\medskip

Finally, in Chapter \ref{cap:futurework}, \textit{Conclusions and future work}, the project's conclusions and potential improvements for the final implementation of the applications that may be of interest are detailed.








