\chapter*{Conclusions and future work}
\label{cap:futurework}
\addcontentsline{toc}{chapter}{Conclusions and future work}

\chapterquote{Do what you think is interesting, do something that you think is fun and worthwhile, because otherwise you won't do it well anyway.}{Brian Kernighan}


\begin{resumenEng}
In this final chapter, the entire work carried out is briefly summarized, also mentioning any work that may be interesting for the future of the project.
\end{resumenEng}

Thanks to the development of \baker, a useful tool has been implemented for creating 2D RPG-style video games. Although it is a project with a smaller scope compared to other engines specialized in the genre, it is capable of generating very complete games that cover many of the requirements these games demand. It brings game creation closer to an audience without programming experience, allowing them to freely create this type of game and opening the doors for a larger number of people to engage in it. One of \baker's strengths is its multiplatform approach, with a functional editor available on Windows, MacOS, and Linux, which facilitates its use in the most common development environments and offers greater convenience to users regardless of their preferred operating system. On the other hand, the engine runs on Windows, MacOS, Linux, and also Android, significantly expanding the reach of published games, especially by including support for one of the most important markets in the current industry: mobile devices.

\medskip

With all this, even having met all our proposed objectives, \baker{} could still offer more tools to add other systems typical of RPGs. After receiving a large amount of user feedback, mainly from users more experienced with tools like \textit{RPG Maker}, the following elements have been identified as possible future expansions:

\begin{itemize}
\item Combat system: Since RPGs commonly involve character battles, adding functionality to create and customize combat scenarios would be a valuable enhancement. Ideally, it would allow developers to define the combatants, set their statistics, determine their attack methods, and configure how they evolve—both for player characters and enemies. The second essential component would be the creation of combat scenes. Given that turn-based combat is a classic element of RPGs, it would fit naturally within this environment. Users could define which characters take part, when each turn’s actions are selected, and in what order the combatants act.

\item Item system: Another staple of RPGs is the use of items, or objects that the player can collect, such as weapons or potions. These would be linked to an inventory system that the player can manage. Items would apply effects that alter character states or game conditions. Each item would have one or more associated effects, which could be applied both during world exploration and in combat, depending on the context.
\end{itemize}

\medskip

In summary, our editor allows users to easily create simple RPGs using the engine, enabling dynamic exploration of a game world. While already a powerful tool, additional features could further enhance the experience and broaden the possibilities for RPG creation. This would allow less experienced developers to build even more complete games.