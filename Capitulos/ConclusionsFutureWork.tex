\chapter*{Conclusions and future work}
\label{cap:futurework}
\addcontentsline{toc}{chapter}{Conclusions and future work}

Thanks to the development of \baker, a useful tool has been implemented for the creation of 2D RPG-style video games. While it may have a smaller scope compared to other engines specialized in the genre, it is capable of producing highly complete games that cover many of the typical requirements. It aims to bring game development closer to users without programming experience, allowing them to freely create this type of game and thereby opening the door for a wider audience to get involved in game creation. Furthermore, since both the editor and the engine are multiplatform, this broadens its accessibility and usability. A multiplatform editor offers greater convenience for creators by adapting more easily to their development environments. Likewise, a multiplatform engine increases the potential reach of the published games, especially by supporting Android, one of the largest markets in the video game industry due to the popularity of mobile devices.

\medskip

Despite having achieved all the proposed objectives, Baker could be expanded to include more tools that support additional systems typically found in RPGs. Some possible future extensions include:

\begin{itemize}
\item Combat system: Since RPGs commonly involve character battles, adding functionality to create and customize combat scenarios would be a valuable enhancement. Ideally, it would allow developers to define the combatants, set their statistics, determine their attack methods, and configure how they evolve—both for player characters and enemies. The second essential component would be the creation of combat scenes. Given that turn-based combat is a classic element of RPGs, it would fit naturally within this environment. Users could define which characters take part, when each turn’s actions are selected, and in what order the combatants act.

\item Item system: Another staple of RPGs is the use of items—objects that the player can collect, such as weapons or potions. These would be linked to an inventory system that the player can manage. Items would apply effects that alter character states or game conditions. Each item would have one or more associated effects, which could be applied both during world exploration and in combat, depending on the context.
\end{itemize}

\medskip

In summary, our editor allows users to easily create simple RPGs using the engine, enabling dynamic exploration of a game world. While already a powerful tool, additional features could further enhance the experience and broaden the possibilities for RPG creation. This would allow less experienced developers to build even more complete games.