\chapter{Videojuegos de Rol}
\label{cap:videojuegosrol}

\begin{resumen}
En este capítulo se explicará qué son los videojuegos, haciendo un enfoque en los videojuegos de rol, se introducirá su historia y se hablará del proceso de desarrollo de estos, analizando la importancia de los motores y editores en el desarrollo de videojuegos.
\end{resumen}

\section{¿Qué es un videojuego?}
Definir qué es un videojuego es una tarea bastante complicada, sobre todo teniendo en cuenta los matices con los que podemos definirlo (académicos, de diseño, experimentales o tecnológicos). Una de las definiciones más acertadas nos la da \cite{EspositoVJ}, que lo define como \comillas{un \textit{juego} que se \textit{juega} gracias a un \textit{aparato audiovisual}, y que puede estar basado en una \textit{historia}}. El propio \citeauthor{EspositoVJ} se encarga de definir qué es el \textit{juego}, qué es \textit{jugar}, qué es el \textit{aparato audiovisual} y qué es la \textit{historia}.

\smallskip

La diferencia fundamental de los juegos tradicionales frente a los videojuegos es la existencia de ese \textit{aparato audiovisual} (videoconsolas, ordenador, dispositivos móviles) que pueda ofrecer una interacción \comillas{humano-máquina}, ya que es esta interacción recíproca la que hace que los videojuegos sean videojuegos y no otro tipo de entretenimiento multimedia.

\subsection{El problema de los géneros de videojuegos}

\subsection{Historia de los videojuegos de rol}

\section{Sobre el desarrollo de videojuegos}

\subsection{Motor de videojuegos}

\subsubsection{Componentes de un motor de videojuegos}

\subsubsection{Separación entre motor y \textit{gameplay}}

\subsubsection{Programación dirigida por datos (DDP) en videojuegos}

\subsubsection{Programación multiplataforma en videojuegos}

\subsection{Editor de videojuegos}

\subsubsection{Editores específicos de videojuegos para desarrollo de RPG}

