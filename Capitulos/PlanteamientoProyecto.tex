\chapter{Planteamiento del Proyecto}
\label{cap:planteamiento}

\begin{resumen}
En este capítulo vamos a hablar de los objetivos principales del proyecto, así como de las decisiones tomadas en cuanto a diseño del motor y del editor.
\end{resumen}

\section{Objetivos principales del Proyecto}
Nuestra idea principal era el desarrollo de un motor de videojuegos, enfocado a los RPG 2D, acompañado de un editor que permitiese un desarrollo rápido y sin mucha complicación de juegos de este tipo, sobre todo para gente no programadora o sin experiencia en el desarrollo de videojuegos. El videojuego generado por el editor debería ser multiplataforma, y poderse ejecutar en Windows, Mac, Linux y Android. Por otra parte, el editor también sería multiplataforma, para las anteriormente mencionadas excepto Android, ya que no es la plataforma más preferible a la hora de poder ejecutar un editor\footnote{Los editores suelen tener elementos que son preferibles de ser utilizados mediante entrada de teclado y ratón. Si bien es cierto que Android permite la conexión de periféricos de entrada-salida, es una plataforma pensada para dispositivos móviles y táctiles.}. 

\medskip

El editor debería ser capaz de poder generar las cuatro versiones del juego distintas para cada plataforma, permitir al usuario elegir para qué plataforma generar el ejecutable, y volcar todo el contenido desarrollado a estos, asegurando que el comportamiento programado fuese el mismo que en el juego final, y que sea esta versión la que el usuario pueda empaquetar y distribuir, sin la necesidad de hacer nada más.

\smallskip

El editor también debe ser capaz de poder generar diversos proyectos (es decir, distintos juegos), y capaz de guardar el estado de un proyecto y poder recuperarlo cuando el usuario desee, sin que este haya podido perder ningún cambio que haya realizado. Y, finalmente, debe poder exportar proyectos, e importar otros que otros usuarios puedan haber generado en otras plataformas sin mayor dificultad.

\medskip

El motor, por su parte, aportará la mayor parte del \textit{gameplay}, para que el usuario solo tenga que desarrollar la parte de diseño (principalmente el diseño artístico y visual. Tendrá que tener soporte para periféricos de entrada-salida tradicionales (teclado y ratón), así como entrada táctil (para los dispositivos móviles). 

\section{Toma de decisiones}
La primera decisión a tomar fue la plataforma de desarrollo del Proyecto. El Proyecto se iba a desarrollar íntegramente en C++, ya que se quería aprovechar la potencia que ofrece con respecto a otros lenguajes, el soporte multiplataforma que tiene la familia C/C++ tanto en dispositivos de sobremesa como en móviles, y el uso de librerías más avanzadas que nos facilitarían a la hora de desarrollar el Proyecto.

\medskip

Debido a nuestra premisa de un desarrollo multiplataforma, se necesitaba usar un IDE (\textit{integrated development environment}, entorno de desarrollo integrado) que fuese compatible tanto con Windows, Mac, y Linux. La opción que en un principio se había valorado era la de utilizar \textit{Visual Studio}, una de las herramientas más populares para el desarrollo en C++; sin embargo, debido a que este IDE carece de versiones para Mac y Linux\footnote{Mac y Linux cuentan con \textit{Visual Studio Code}, que si bien sirve para poder compilar C/C++, es más complicado de configurar para proyectos más complejos como este.}, se optó por hacer el desarrollo del Proyecto en \textit{CLion}, un IDE con soporte para CMake, que facilitaría a la hora de agilizar el trabajo (la compilación utilizando CMake, en nuestra experiencia, es considerablemente más rápida que la compilación utilizando el compilador por defecto de \textit{Visual Studio}) y con la gestión de las dependencias externas.

\smallskip

Pese a que el aprender a usar CMake ocupó gran parte del inicio del desarrollo del Proyecto, las ventajas que han supuesto a la hora del manejo de las distintas dependencias externas frente a la alternativa que se había valorado\footnote{La alternativa era utilizar \textit{submódulos} en GitHub, pero eso suponía un elevado coste extra a la hora de clonar el proyecto (en lugar de clonar únicamente el código a una máquina, todas las librerías que se utilizasen tendrían que ser clonadas), actualizar las librerías, etc\ldots} han hecho que esta opción haya merecido la pena.

\medskip

Por otra parte, se tendría que utilizar otra herramienta para el desarrollo de la APK (\textit{Android Application Package}, paquete de aplicaciones Android, es decir, el \comillas{ejecutable} de Android), ya que esta se debe desarrollar utilizando Java, por lo que se decidió utilizar \textit{Android Studio}, la herramienta oficial de desarrollo para Android, que proporciona máquinas virtuales de dispositivos Android de distintas versiones y generaciones para poder probar la \textit{build}.

\smallskip

Dentro de \textit{Android Studio}, se tendría que añadir también el módulo de NDK (\textit{Native Development Kit}, kit de desarrollo nativo) que permite el desarrollo de aplicaciones para Android utilizando llamadas a C/C++ gracias a JNI (\textit{Java Native Interface}, interfaz nativa de Java), una FFI (\textit{foreign function interface}, interfaz de funciones foráneas, es decir, un mecanismo por el cual un lenguaje de programación puede llamar a funciones o rutinas programadas o compiladas en otro lenguaje de programación distinto) integrada en el SDK de Java que permite la comunicación con C/C++. Esto sería necesario para poder ejecutar el juego, pese a que la entrada de la aplicación estuviese en Java.

\medskip

El resto de decisiones son propias de cada una de las partes del Proyecto y vamos a detallarlas a continuación.

\subsection{Diseño del motor}
En cuanto al diseño del motor, las primeras decisiones giraban en torno al contenido \textit{gameplay} que se quería que este tuviese. La principal idea era poder diseñar juegos al estilo de las primeras generaciones de la saga \cite{pokemon} (es decir, un juego \textit{pokemonlike}). Finalmente, se determinó que el contenido fuese el siguiente:
\begin{itemize}
	\item Mapas, esenciales para el desarrollo de videojuegos. Estos estarían formados por dos elementos, un \textit{tilemap} (es decir, un conjunto de baldosas individuales unidas en un mismo archivo) y objetos.
	\item Personajes, con características personalizables y que pudiesen interactuar con la escena mediante el uso de eventos.
	\item Objetos, que son elementos que contienen atributos, como por ejemplo \textit{sprites}, eventos, sonidos, etc\ldots
\end{itemize}

El motor seguiría un patrón EC (entidad-componente), con las entidades agrupadas en distintas escenas.

\subsection{Diseño del editor}

