\chapter*{Resumen}

\section*{\tituloPortadaVal}

Los videojuegos de rol (\textit{RPG}, \textit{role-playing game}) han experimentado grandes cambios desde su aparición, a mediados de la década de los 70, donde, por las limitaciones tecnológicas de la época, no eran más que adaptaciones de juegos de mesa ya existentes, como \textit{Dungeons \& Dragons}; hasta la actualidad, donde se han elaborado multitud de aventuras y temáticas originales y se han convertido en uno de los géneros más populares en el mercado. 
\\

%Es en esta década también cuando vemos dos tendencias claramente diferenciadas: por una parte, los videojuegos de rol <<\textit{occidentales}>> (\textit{WRPG}, \textit{Western role-playing game}), con más libertad de decisión para los jugadores tanto en personalización como en la historia, y con temáticas principalmente realistas; y, los videojuegos de rol <<\textit{orientales}>> o <<\textit{japoneses}>> (\textit{JRPG}, \textit{Japanese role-playing game}), mucho más centrados en una narrativa lineal con temáticas fantásticas y con mecánicas basadas por turnos. Estos últimos se desarrollaron prácticamente en Japón hasta la entrada del nuevo milenio (aunque también hay \textit{RPG} desarrollados en Corea del Sur de la década de los 80 y 90), de ahí el nombre que reciben actualmente.


\section*{Palabras clave}
   
\noindent Videojuegos de Rol, Desarrollo de Videojuegos, Editor de Videojuegos, Herramienta de Desarrollo,

   


