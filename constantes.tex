%---------------------------------------------------------------------
%
%                          constantes.tex
%
%---------------------------------------------------------------------
%
% Fichero que  declara nuevos comandos LaTeX  sencillos realizados por
% comodidad en la escritura de determinadas palabras
%
%---------------------------------------------------------------------

%%%%%%%%%%%%%%%%%%%%%%%%%%%%%%%%%%%%%%%%%%%%%%%%%%%%%%%%%%%%%%%%%%%%%%
% Comando: 
%
%       \titulo
%
% Resultado: 
%
% Escribe el título del documento.
%%%%%%%%%%%%%%%%%%%%%%%%%%%%%%%%%%%%%%%%%%%%%%%%%%%%%%%%%%%%%%%%%%%%%%
\def\titulo{Motor y editor de videojuegos en 2D enfocado al desarrollo de juegos RPG}

%%%%%%%%%%%%%%%%%%%%%%%%%%%%%%%%%%%%%%%%%%%%%%%%%%%%%%%%%%%%%%%%%%%%%%
% Comando: 
%
%       \autor
%
% Resultado: 
%
% Escribe el autor del documento.
%%%%%%%%%%%%%%%%%%%%%%%%%%%%%%%%%%%%%%%%%%%%%%%%%%%%%%%%%%%%%%%%%%%%%%
\def\autor{Miguel Curros Garc\'ia, Alejandro Gonz\'alez S\'anchez y Alejandro Mass\'o Mart\'inez}

% Variable local para emacs, para  que encuentre el fichero maestro de
% compilación y funcionen mejor algunas teclas rápidas de AucTeX

%%%
%%% Local Variables:
%%% mode: latex
%%% TeX-master: "tesis.tex"
%%% End:

%%%%%%%%%%%%%%%%%%%%%%%%%%%%%%%%%%%%%%%%%%%%%%%%%%%%%%%%%%%%%%%%%%%%%%
% Comando: 
%
%       \citegame
%
% Resultado: 
%
% Cita un juego
%%%%%%%%%%%%%%%%%%%%%%%%%%%%%%%%%%%%%%%%%%%%%%%%%%%%%%%%%%%%%%%%%%%%%%
\newcommand{\citegame}[2]{\textnormal{\cite{#2}}}

%%%%%%%%%%%%%%%%%%%%%%%%%%%%%%%%%%%%%%%%%%%%%%%%%%%%%%%%%%%%%%%%%%%%%%
% Comando: 
%
%       \comillas
%
% Resultado: 
%
% Entrecomilla un texto usando << y >>
%%%%%%%%%%%%%%%%%%%%%%%%%%%%%%%%%%%%%%%%%%%%%%%%%%%%%%%%%%%%%%%%%%%%%%
\newcommand{\comillas}[1]{\guillemotleft #1\guillemotright{}}

%%%%%%%%%%%%%%%%%%%%%%%%%%%%%%%%%%%%%%%%%%%%%%%%%%%%%%%%%%%%%%%%%%%%%%
% Comando: 
%
%       \figuratikz
%
% Resultado: 
%
% Genera una figura utilizando el entorno de tikz.
% Parámetros: Texto con descrpción, etiqueta de referencia, código tikz
%%%%%%%%%%%%%%%%%%%%%%%%%%%%%%%%%%%%%%%%%%%%%%%%%%%%%%%%%%%%%%%%%%%%%%
\newcommand{\figuratikz}[3]{%
\begin{figure}[t]%
\begin{center}%
\begin{tikzpicture}%
	#3
\end{tikzpicture}%
\caption{#1} \label{#2}%
\end{center}%
\end{figure}%
}

%%%%%%%%%%%%%%%%%%%%%%%%%%%%%%%%%%%%%%%%%%%%%%%%%%%%%%%%%%%%%%%%%%%%%%
% Comando: 
%
%       \figuratikz
%
% Resultado: 
%
% Genera una figura utilizando el entorno de tikz.
% Parámetros: Texto con descrpción, etiqueta de referencia, argumentos tikz,
% código tikz
%%%%%%%%%%%%%%%%%%%%%%%%%%%%%%%%%%%%%%%%%%%%%%%%%%%%%%%%%%%%%%%%%%%%%%
\newcommand{\figuratikzparametros}[4]{%
\begin{figure}[t]%
\begin{center}%
\begin{tikzpicture}[#3]%
	#4
\end{tikzpicture}
\caption{#1} \label{#2}%
\end{center}%
\end{figure}%
}

\newcommand{\formatdate}[1]{%
  \DTMdisplaydate{%
    \DTMfetchyear{#1}}{%
    \DTMfetchmonth{#1}}{%
    \DTMfetchday{#1}}{%
    -1}}
    
\newcommand{\baker}{%
	\textsc{RPGBaker}%
}

\newcommand{\figuraH}[2]{%
\begin{center}%
\imagen{#1}{#2}%
\end{center}%
}